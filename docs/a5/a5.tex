\documentclass[12pt, titlepage]{article}
%Packages
\usepackage{
  acronym,
  enumerate,
  fancyhdr,
  hyperref,
  indentfirst,
  lastpage,
  listings,
  microtype,
}
\usepackage[top=1in, bottom=1in, left=1in, right=1in]{geometry}
\usepackage[htt]{hyphenat}

% declare some useful macros
\newcommand{\assignmentNumber}{Assignment 5}
\newcommand{\courseName}{CS 444}
\newcommand{\z}[1]{\texttt{#1}}

\acrodef{jls}[JLS]{Java Language Specification}
\acrodef{ir}[IR]{Intermediate Representation}
\acrodef{dfs}[DFS]{Depth-First Search}

\lstset{basicstyle=\ttfamily}

\begin{document}
\pagestyle{fancyplain}
\thispagestyle{plain}

%Headers
\fancyhead{}
\fancyfoot{}
\rhead{\fancyplain{}{Page \thepage\ of \pageref*{LastPage}}}
\chead{\fancyplain{}{\assignmentNumber}}
\lhead{\fancyplain{}{\courseName}}

\title{CS444 - \assignmentNumber\\Code Generation}
\date{\today}
\author{Alex Klen\\20372654\and Sanjay Menakuru\\20374915\and Jonathan Wei\\20376489}

\maketitle

\section{Introduction}

In this assignment, we built a backend for our compiler that emits 32-bit
\z{x86} assembly. In an effort to make the majority of our compiler portable to
other architectures, we split the code generation into two parts: an
architecture-independent intermediate representation, and an i386-specific
backend. We also wrote a small portion of the runtime in Joos itself. In this
document, we will discuss these three components, the challenges we faced, and
our testing strategies.

\section{Architecture}\label{sec:arch}

\subsection{\ac{ir}}\label{subsec:ir}

We wanted our compiler to be able to generalize to different target
architectures. To this end, we created a Joos-specific, but
platform-independent \ac{ir}. Our core abstraction is a \z{Stream}, which can
be found in \z{ir/stream.h}. A \z{Stream} contains the \ac{ir} for a single
Joos method. We have a \z{Visitor} implementation that walks over the AST
output from our frontend and generates a \z{Stream} for every method in the
program. This \z{Visitor} can be found in \z{ir/ir\_generator.\{h,cpp\}}.
Notably, it also generates some virtual \z{Stream}s that do not correspond to
methods written by the user. An example is the \z{\_static\_init} function
which initializes all static fields in the program. 

Instead of directly creating a \z{Stream}, the \z{Visitor} makes use of a
\z{StreamBuilder}, which can be found in \z{ir/stream\_builder.\{h,cpp\}}. The
\z{StreamBuilder} validates that the generated \ac{ir} is free of many common
errors. For example, it verifies that no memory is read before it is written.
This abstraction was extremely helpful when developing the \ac{ir}-generating
\z{Visitor}, as it caught many programmer-introduced errors.

The \ac{ir} instructions operate on \z{Mem}s. A \z{Mem} is a conceptual memory
address that can be read and written. Each \z{Mem} has an associated
\z{SizeClass}; these range from \z{BOOL} to \z{PTR}. The bit-width of the
\z{PTR} \z{SizeClass} is architecture-dependent; all other \z{SizeClass}es have
intrinsic bit-widths defined by the \ac{jls}.

The \z{Mem} class' public API is defined in \z{ir/mem.\{h,cpp\}}. Its
implementation is in \\\z{ir/mem\_impl.\{h,cpp\}}. The \z{SizeClass}es are
defined and implemented in \z{ir/size.\{h,cpp\}}.

An example of an \ac{ir} instruction is \z{ADD}. \z{ADD} takes 3 \z{Mem}s as
arguments: a destination \z{Mem}, and two operand \z{Mem}s.

\subsection{Backend}\label{subsec:backend}

The \z{backend} directory houses all code responsible for lowering \ac{ir} to
assembly. It is split into two further subdirectories: \z{common} and \z{i386}.

The \z{backend/common} subdirectory contains the platform-independent
computations necessary for generating Joos binaries. For instance, the code to
generate vtable-offsets lives in \z{backend/common}.

The \z{backend/i386} subdirectory contains code that generates 32-bit \z{x86}
assembly from \ac{ir}.

We will now discuss these two subdirectories in more detail.

\subsubsection{Common}\label{subsubsec:common}

The main utility of interest in the \z{backend/common} directory is the
\z{OffsetTable} class, which can be found in
\z{backend/common/offset\_table.\{h,cpp\}}. This class inspects the data
structures output by our typechecker and builds several lookup tables that are
useful during code generation. These tables include the following:
\begin{enumerate}
  \item A mapping between types and their sizes.
  \item A mapping between instance fields and their offsets inside their
    containing class.
  \item A mapping between instance methods and their offsets inside their
    containing class' vtable. See the \hyperref[subsubsec:vtables]{Vtables}
    section for details.
  \item A mapping between interface methods and their offsets inside the
    itables. See the \hyperref[subsubsec:itables]{Itables} section for details.
  \item A mapping between types and their corresponding static fields.
  \item A mapping between native methods and their corresponding
    fully-qualified labels.
\end{enumerate}

Another utility in \z{backend/common} is the \z{AsmWriter} struct, located in
\\\z{backend/common/asm\_writer.h}. It exposes a simple API that enables writing
nicely-indented assembly. It makes use of a custom \z{Fprintf} implementation
located in \z{base/printf.h}. Unlike the \z{fprintf} found in \z{libc}, our
implementation uses \z{C++11}'s variadic templates to guarantee type and memory
safety.

\subsubsection{\z{i386}}\label{subsubsec:i386}

The only class in the \z{backend/i386} directory is the \z{Writer} class, which
is located in \z{backend/i386/writer.\{h,cpp\}}. It is a large class that
handles lowering from our \ac{ir} to assembly. It is also responsible for
writing several runtime data structures. For instance, it writes all constant
static strings into the resulting assembly as an alternating series of
\z{String} headers and character arrays.

\subsection{Runtime}\label{subsec:runtime}

\subsubsection{Vtables}\label{subsubsec:vtables}
We used the standard mechanism for polymorphic method dispatch, commonly
referred to as `vtables'. The vtables themselves are written into the
\z{.rodata} segment of the program. The code to generate these tables can be
found in \z{Writer::WriteVtableImpl}.

\subsubsection{Itables}\label{subsubsec:itables}
We used a slightly non-standard implementation of interface method dispatch. In
particular, the standard approach involves having a `selector index' in the
vtable which is an index into a global table of function pointers. In our case,
we placed a pointer to the relevant column of the global table into the vtable
directly. This allowed us to distribute the global table amongst each type;
this was easier to debug, and generated smaller and more readable assembly.

\subsubsection{Type coercion}\label{subsubsec:co}
We used a highly non-standard implementation of type-coercion. This is the
runtime-portion of the \z{instanceof} expression and the cast expression.

We were unsatisfied with the quadratic space complexity of the standard
approach. Instead, we encoded the object hierarchy into a series of arrays, and
wrote Joos code that implemented a memoized \ac{dfs} of the object hierarchy.
The memoization still could require quadratic space, but this space is lazily
allocated on demand from the heap, rather than pessimistically reserving space
in the binary itself. Note that this approach has the downside of requiring
allocation during the first \z{instanceof} or cast usage for a given type. This
approach is therefore unsuitable for environments with strict bounds on heap
allocation, such as embedded domains.

The Joos code for implementing the memoized \ac{dfs} can be found in
\\\z{runtime/\_\_joos\_internal\_\_/TypeInfo.java}. This class (and others
listed below) are automatically compiled into every binary produced by our
compiler.

Instances of the \z{TypeInfo} class are stored within the vtables of each
class. These instances are consulted during the evaluation of all
\z{instanceof} and cast expressions.

\subsubsection{\z{String} operations}\label{subsubsec:strops}
The \z{String} concatenation operation supports primitive operands. Per the
\ac{jls}, we must first convert primitives to references; we do this by calling
the \z{String.valueOf} overload that accepts the given primitive type.

After performing the conversion above, we are guaranteed that both operands are
reference types. We then call the \z{StringOps.Str} method, which can be found
in \\\z{runtime/\_\_joos\_internal\_\_/StringOps.java}. This method implements
the algorithm specified in the \ac{jls} for converting arbitrary references to
a non-null \z{String}.

After the step above, we are guaranteed that both operands are non-null
\z{String}s. Finally, we de-sugar the concatenation itself to call the
\z{String.concat} instance method. Specifically, we generate
\z{lhs.Concat(rhs)}.

\subsubsection{Stack frames}\label{subsubsec:stacks}
In the event that user code performs an illegal operation, our generated binary
will print a stack trace before calling the provided \z{\_\_exception}
procedure. These stack traces were invaluable in debugging our runtime.

This is accomplished by using the \z{StackFrame} class, which can be found in
\\\z{runtime/\_\_joos\_internal\_\_/StackFrame.java}. For every call site in
the program, we generate a static instance of the \z{StackFrame} class. When
performing a call, we push the corresponding static instance onto the stack,
and subsequently pop it off the stack when the call returns.

In the event of an exception, we walk the stack and use the \z{StackFrame}
instances to print out the corresponding stack trace.

\section{Files}
Here is a mapping from various files added for this assignment to the relevant
sections discussing their functionality.

\begin{itemize}
  \item \z{ir/stream.h}: See \hyperref[subsec:ir]{\acf{ir}}
  \item \z{ir/size.\{h, cpp\}}: See \hyperref[subsec:ir]{\acf{ir}}
  \item \z{ir/stream\_builder.\{h, cpp\}}: See \hyperref[subsec:ir]{\acf{ir}}
  \item \z{ir/ir\_generator.\{h, cpp\}}: See \hyperref[subsec:ir]{\acf{ir}}
  \item \z{ir/mem\_impl.\{h, cpp\}}: See \hyperref[subsec:ir]{\acf{ir}}
  \item \z{ir/mem.\{h, cpp\}}: See \hyperref[subsec:ir]{\acf{ir}}
  \item \z{backend/i386/writer.\{h, cpp\}}: See \hyperref[subsubsec:i386]{i386}
  \item \z{backend/common/offset\_table.\{h, cpp\}}: See \hyperref[subsubsec:common]{Common}
  \item \z{backend/common/asm\_writer.h}: See \hyperref[subsubsec:common]{Common}
  \item \z{runtime/\_\_joos\_internal\_\_/TypeInfo.java}: See \hyperref[subsubsec:co]{Type coercion}
  \item \z{runtime/\_\_joos\_internal\_\_/StringOps.java}: See \hyperref[subsubsec:strops]{String operations}
  \item \z{runtime/\_\_joos\_internal\_\_/StackFrame.java}: See \hyperref[subsubsec:stacks]{Stack frames}
  \item \z{runtime/\_\_joos\_internal\_\_/Array.java}: See \hyperref[subsec:arrays]{Arrays}
  \item \z{runtime/runtime.\{h, cpp\}}: See \hyperref[subsec:runtime]{Runtime}
\end{itemize}

\section{Challenges}
\subsection{Arrays}\label{subsec:arrays}
Arrays were particularly challenging to implement. First, type coercion between
arrays and non-arrays runs a particularly complicated algorithm that was
difficult to define. The algorithm provided in the \ac{jls} handled many more
cases than we needed. In particular, it supported multi-dimensional arrays and
type-introspection. Since Joos supports neither of these features, we decided
to write our own simplified version of the algorithm. This proved to be quite
difficult, and had many edge cases. For instance, we needed to allow a cast
from an array to any of \z{Object}, \z{Cloneable}, or \z{Serializable}. To
accomplish this, we wrote an internal interface named \z{Array}, which can be
found in \\\z{runtime/\_\_joos\_internal\_\_/Array.java}. At runtime, every
array implements this interface. This makes our standard type coercion
algorithm work with array to non-array casts.

Another challenge with arrays was their polymorphic nature. That is to say, all
stores to an array must be assignable to the dynamic element type of the array.
This is difficult because Java treats arrays covariantly, so this check must be
performed at runtime. We solved this by giving each array two \z{TypeInfo}
pointers: the first points to the \z{TypeInfo} for the array itself, and the
second points to the \z{TypeInfo} for the elements. On every store to an array
of references, we ensure that the types are assignable.

\subsection{Debugging generated assembly}
Our type checker performed `alpha-renaming'. This meant that all symbols were
renamed to globally unique compiler-assigned integers, rather than
user-specified strings. However, this resulted in rather inscrutable assembly.
For example, the method \z{java.lang.String.valueOf} might be converted to
\z{\_t20\_m56}.

We made an effort to output many comments in the generated assembly to help
track the flow of data through registers and the stack. Even so, our backend's
fondness for generating many temporary variables made it tricky to track what
went where. We wrote a prototype registerizer that preferred storing
temporaries in registers. However, we found it tricky to maintain and so never
merged into our mainline repository.

\section{Testing}

We switched our build system from a complex homegrown Makefile to use Google's
recently open-sourced build system, `Bazel'. Our Makefile had support for
compilation-parallelism, but ran all of our tests in serial. Bazel supports
both compilation-parallelism, as well as test-parallelism. This significantly
dropped the time to run all of our tests. Further, Bazel aggressively caches
builds and test outputs to avoid redundant work. For instance, since we never
modified our lexer, Bazel would run the tests for the lexer once, and never run
them again as the test binary had not changed.

This made our development cycle faster, and was a significant improvement in
our productivity. We had wrote custom Bazel targets that ran all Marmoset tests
locally. To run all of the tests in our project (including Marmoset tests) we
simply ran \z{bazel test :all}. Previously, running all such tests up to A4
took roughly 50 seconds. Bazel takes 30 seconds to \emph{compile} and run all
tests up to A5. Note that for the purpose of building the binary on Marmoset,
our original is still included. However, it won't be able to run all of our
tests.

\end{document}

